% ================================== HEADER ====================================
\documentclass{article}           % Sets style/look of many things.
% \documentclass{report}          % part, chapters, front page etc.
\usepackage{exsheets}
\usepackage[utf8]{inputenc}       % Encoding of input files UTF-8
\usepackage[T1]{fontenc}
\usepackage[scaled]{beramono}     % Font
\usepackage{color}                % Color text
\usepackage{titlesec}             % Select alternative section titles
\usepackage{fancyvrb}
\usepackage{verbatim}             % Comment environment
\usepackage{listings}             % Format and render text/code etc.
\usepackage{minted}               % Much better syntax highlighting
\usepackage{float}                % Control of floating environment/figure
\usepackage{graphicx,  subfigure} % Better figures, graphics, units etc.
\usepackage{multicol}             % Multiple columns
\usepackage{amsmath}              % Math: Equation, split, align etc.
\usepackage{siunitx}              % SI units
\usepackage{mathtools}            % Different math tools to use with amsmath
\usepackage{amssymb}              % Math symbols
\usepackage[
    colorlinks,
    citecolor=black,              % I like links with standard black color
    filecolor=black,
    linkcolor=black,
    urlcolor=black
]{hyperref}                       % Links in TOC etc.
\usepackage[all]{hypcap}          % Better links to floating environment

\usepackage{tabto}
\newcommand\marginsymbol[1][0pt]{%
  \tabto*{0cm}\makebox[\dimexpr-1cm-#1\relax][r]{$\mathbb{P}$}\tabto*{\TabPrevPos}}

\renewcommand{\thesubsection}{\thesection.\alph{subsection}}
\title{\vspace{-2cm}INF3490/INF4490 Exercises - Week 4}
\author{Ole Herman Schumacher Elgesem, Magnus Olden, Stian Petlund}
\date{\today}

% Removing paragraph indents is sometimes useful:
\setlength\parindent{0pt}

% Make margins smaller to fit more figures, tables etc on page: (optional)
\addtolength{\oddsidemargin}{-1.0in}
\addtolength{\evensidemargin}{-1.0in}
\addtolength{\textwidth}{2.0in}
\addtolength{\topmargin}{-0.8in}
\addtolength{\textheight}{1.6in}
% ==============================================================================

% ================================= DOCUMENT ===================================
\begin{document}
    \renewcommand\marginsymbol[1][0pt]{%
  \tabto*{0cm}\makebox[-1cm][c]{$\mathbb{P}$}\tabto*{\TabPrevPos}}

\maketitle
\(\mathbb{P}\) marks the programming exercises, we strongly recommend using
the python programming language for these. Exercises may be added/changed
after publishing.

\section{Evolution strategy(ES)} % 1
\subsection{} % a
A common variant of evolution strategies used for (local) search is the \((1 + 4)\) ES.
How would this differ from the \((1 + 1)\) ES in how the search space is explored?
How does this, and \((1 + \lambda)\) in general, compared to hill climbing and greedy search?
\subsection{} % b
What effect does an adaptive search strategy have on optimization performance?
\subsection{} % c
How would it affect the search if the strategy parameters were mutated after the solution parameters instead of before?
\section{ES Implementation} % 2
\subsection{\marginsymbol} % a
\label{subsec:w4e3a}
Ignoring mutation, and starting with the population \(\{1, 2, 3, 4\}\),
implement and run 3 generations of a \((4 + 8)\) ES maximizing \(g(x) = x\), and observe what the end population looks like (use intermediary recombination).
\subsection{} % b
If a \((4, 8)\) ES had been used in Problem \ref{subsec:w4e3a}, what would the probability of the optimal solution \((x = 4)\) surviving the first generation have been?
\subsection{\marginsymbol} % c
Repeat Problem \ref{subsec:w4e3a} with an EP with \(q = 2\). How do the two algorithms compare?
\section{Knapsack problem}
In a 0-1 \href{https://en.wikipedia.org/wiki/Knapsack_problem}{\textbf{knapsack problem}}, how could you implement a repair mutation to transform infeasible solutions into feasible ones
(i.e. make the sum of costs of the selected items go below the budget)?
\section*{Corrections and suggestions}
Corrections of grammar, language, notation or suggestions for improving these exercises are appreciated.
E-mail me at: \href{mailto:olehelg@uio.no}{\textbf{olehelg@uio.no}} or use
\href{https://github.com/olehermanse/INF3490-AI_Machine_Learning}{\textbf{GitHub}}
to submit an issue or create a pull request.
\end{document}
% ==============================================================================
