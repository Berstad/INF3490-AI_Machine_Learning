% ================================== HEADER ====================================
\documentclass{article}           % Sets style/look of many things.
% \documentclass{report}          % part, chapters, front page etc.
\usepackage{exsheets}
\usepackage[utf8]{inputenc}       % Encoding of input files UTF-8
\usepackage[T1]{fontenc}
\usepackage[scaled]{beramono}     % Font
\usepackage{color}                % Color text
\usepackage{titlesec}             % Select alternative section titles
\usepackage{fancyvrb}
\usepackage{verbatim}             % Comment environment
\usepackage{listings}             % Format and render text/code etc.
\usepackage{minted}               % Much better syntax highlighting
\usepackage{float}                % Control of floating environment/figure
\usepackage{graphicx,  subfigure} % Better figures, graphics, units etc.
\usepackage{multicol}             % Multiple columns
\usepackage{amsmath}              % Math: Equation, split, align etc.
\usepackage{siunitx}              % SI units
\usepackage{mathtools}            % Different math tools to use with amsmath
\usepackage{amssymb}              % Math symbols
\usepackage[
    colorlinks,
    citecolor=black,              % I like links with standard black color
    filecolor=black,
    linkcolor=black,
    urlcolor=black
]{hyperref}                       % Links in TOC etc.
\usepackage[all]{hypcap}          % Better links to floating environment

\usepackage{tabto}
\newcommand\marginsymbol[1][0pt]{%
  \tabto*{0cm}\makebox[\dimexpr-1cm-#1\relax][r]{$\mathbb{P}$}\tabto*{\TabPrevPos}}

\renewcommand{\thesubsection}{\thesection.\alph{subsection}}
\title{\vspace{-2cm}INF3490/INF4490 Exercise Solutions - Week 2}
\author{Ole Herman S. Elgesem}
\date{\today}

% Removing paragraph indents is sometimes useful:
\setlength\parindent{0pt}

% Make margins smaller to fit more figures, tables etc on page: (optional)
\addtolength{\oddsidemargin}{-1.0in}
\addtolength{\evensidemargin}{-1.0in}
\addtolength{\textwidth}{2.0in}
\addtolength{\topmargin}{-0.8in}
\addtolength{\textheight}{1.6in}
% ==============================================================================

% ================================= DOCUMENT ===================================
\begin{document}
    \renewcommand\marginsymbol[1][0pt]{%
  \tabto*{0cm}\makebox[-1cm][c]{$\mathbb{P}$}\tabto*{\TabPrevPos}}

\maketitle
\(\mathbb{P}\) marks the programming exercises, we strongly recommend using
the python programming language for these. Exercises may be added/changed
after publishing.

\section{Representations}
Recall all the representations that have been presented. Which mutation and
recombination operators are compatible with which representations?\\

\emph{Answer:}

\begin{itemize}
    \item Binary representation
    \begin{itemize}
        \item Bit-flip mutation
        \item N-point and uniform crossover
    \end{itemize}
    \item Integer representation
    \begin{itemize}
        \item Random reset and creep mutation
        \item N-point and uniform crossover
    \end{itemize}
    \item Cardinal/enumerated/symbolic representations
    \begin{itemize}
        \item Random reset mutation
        \item N-point and uniform crossover
    \end{itemize}
    \item Real-valued/Continuous representation
    \begin{itemize}
        \item Uniform and Gaussian mutation
        \item N-point, discrete uniform and arithmetic crossover
    \end{itemize}
    \item Permutation representation
    \begin{itemize}
        \item Swap, insert, scramble and invert mutation
        \item Partially mapped, order, cycle and edge crossover
    \end{itemize}
    \item Tree representation
    \begin{itemize}
        \item Mutation by random replacement
        \item Subtree swap mutation
    \end{itemize}
\end{itemize}

\section{Bit flip mutation}
Given the binary chromosome with length 4, calculate the probability that no
bits, one bit and more than one bit will be flipped in a bit-flip mutation with
\(p_m = \frac{1}{4}\).\\

\emph{Answer:}

Probability of no mutation:
\begin{equation*}
    P(0) = \frac{3}{4} \frac{3}{4} \frac{3}{4} \frac{3}{4} = \frac{3^4}{4^4} \approx 32\%
\end{equation*}
\href{https://en.wikipedia.org/wiki/Binomial_distribution}{Binomial probability}:
\begin{equation}
    P = {n \choose k} p^k (1-p)^{n-k}
\end{equation}
\href{https://en.wikipedia.org/wiki/Binomial_coefficient}{Binomial coefficient}:
\begin{equation}
    {n \choose k} = \frac{n!}{k!(n-k)!}
\end{equation}
Where \(n\) is number of events, \(k\) is number of occurences wanted and \(p\) is the probability of a single event.\\

In this case we want exactly 1 mutation in 4 events, and the probability is 0.25:
\begin{align*}
    n &= 4\\
    k &= 1\\
    p &= p_m = 0.25\\
    P(1) &= {4 \choose 1} 0.25^1 (1-0.25)^{4-1}\\
    P(1) &= \frac{4!}{1!(4-1)!} 0.25^1 (1-0.25)^{4-1}\\
    P(1) &= \frac{4*3*2}{3*2} 0.25^1 (0.75)^{3}\\
    P(1) &= 4 * 0.25^1 (0.75)^{3}\\
    P(1) &= 0.75^3 \approx 42\%
\end{align*}
This makes intuitive sense and we could also arrive at this result without using the general formula.
For example, the probability of mutation (yes,no,no,no) is:
\begin{align*}
    P(1) &= \frac{1}{4} \frac{3}{4} \frac{3}{4} \frac{3}{4}\\
    P(1) &= 0.25 * 0.75^3
\end{align*}
And there are 4 variants, 4 places where the mutation can happen, so:
\begin{align*}
    P(1) &= 4 * (0.25 * 0.75^3)\\
    P(1) &= 0.75^3 \approx 42\%\\
\end{align*}
Finally, the probability of more than one mutation:
\begin{align*}
    P(2+) &= 1 - P(0) - P(1)\\
    P(2+) &\approx 1 - 32\% - 42\%\\
    P(2+) &\approx 26\%
\end{align*}

\section{Crossover \marginsymbol}
Given the sequences (2,4,7,1,3,6,8,9,5) and (5,9,8,6,2,4,1,3,7). Implement
these algorithms to create a new pair of solutions:
\renewcommand{\theenumi}{\alph{enumi}}
\begin{enumerate}
  \item Partially mapped crossover (PMX).
  \item Order crossover.
  \item Cycle crossover.
\end{enumerate}

\emph{Answer:}

\subsection{Partially mapped crossover}
\subsubsection{Output}
\begin{verbatim}
Parents:
[2, 4, 7, 1, 3, 6, 8, 9, 5]
[5, 9, 8, 6, 2, 4, 1, 3, 7]
Children:
[5, 9, 7, 1, 3, 6, 4, 2, 8]
[3, 1, 8, 6, 2, 4, 7, 9, 5]
\end{verbatim}
\subsubsection{Source code}
\inputminted{Python}{pmx.py}
\subsection{Order Crossover}
\subsubsection{Output}
\begin{verbatim}
Parents:
[2, 4, 7, 1, 3, 6, 8, 9, 5]
[5, 9, 8, 6, 2, 4, 1, 3, 7]
Children:
[9, 2, 4, 1, 3, 6, 8, 7, 5]
[7, 3, 8, 6, 2, 4, 1, 9, 5]
\end{verbatim}
\subsubsection{Source code}
\inputminted{Python}{order.py}
\subsection{Cycle Crossover}
\subsubsection{Output}
\begin{verbatim}
Parents:
[2, 4, 7, 1, 3, 6, 8, 9, 5]
[5, 9, 8, 6, 2, 4, 1, 3, 7]
Children:
[2, 4, 7, 1, 3, 6, 8, 9, 5]
[5, 9, 8, 6, 2, 4, 1, 3, 7]
\end{verbatim}
(These 2 parents have just one cycle, so the child will be the same as parent 1).
\subsubsection{Source code}
\inputminted{Python}{cycle.py}

\section*{Contact and Github}
Corrections of grammar, language, notation or suggestions for improving this material are appreciated.
E-mail me at \href{mailto:olehelg@uio.no}{\textbf{olehelg@uio.no}} or use \href{https://github.com/olehermanse/INF3490-AI_Machine_Learning}{\textbf{GitHub}} to submit an issue or create a pull request.
The \href{https://github.com/olehermanse/INF3490-AI_Machine_Learning}{\textbf{GitHub repository}} contains all source code for assignments, exercises, solutions, examples etc.
As many people have been involved with writing and updating the course material, they are not all listed as authors here.
For a more complete list of authors and contributors see the \href{https://github.com/olehermanse/INF3490-AI_Machine_Learning/blob/master/README.md}{\textbf{README}}.

\end{document}
% ==============================================================================
