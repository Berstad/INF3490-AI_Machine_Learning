% ================================== HEADER ====================================
\documentclass{article}           % Sets style/look of many things.
% \documentclass{report}          % part, chapters, front page etc.
\usepackage{exsheets}
\usepackage[utf8]{inputenc}       % Encoding of input files UTF-8
\usepackage[T1]{fontenc}
\usepackage[scaled]{beramono}     % Font
\usepackage{color}                % Color text
\usepackage{titlesec}             % Select alternative section titles
\usepackage{fancyvrb}
\usepackage{verbatim}             % Comment environment
\usepackage{listings}             % Format and render text/code etc.
\usepackage{minted}               % Much better syntax highlighting
\usepackage{float}                % Control of floating environment/figure
\usepackage{graphicx,  subfigure} % Better figures, graphics, units etc.
\usepackage{multicol}             % Multiple columns
\usepackage{amsmath}              % Math: Equation, split, align etc.
\usepackage{siunitx}              % SI units
\usepackage{mathtools}            % Different math tools to use with amsmath
\usepackage{amssymb}              % Math symbols
\usepackage[
    colorlinks,
    citecolor=black,              % I like links with standard black color
    filecolor=black,
    linkcolor=black,
    urlcolor=black
]{hyperref}                       % Links in TOC etc.
\usepackage[all]{hypcap}          % Better links to floating environment

\usepackage{tabto}
\newcommand\marginsymbol[1][0pt]{%
    \tabto*{0cm}\makebox[\dimexpr-1cm-#1\relax][r]{$\mathbb{P}$}\tabto*{\TabPrevPos}}

\renewcommand{\thesubsection}{\thesection.\alph{subsection}}
\title{\vspace{-2cm}INF3490/INF4490 Exercises - Week 1}
\author{Ole Herman Schumacher Elgesem, Magnus Olden, Stian Petlund}
\date{\today}

% Removing paragraph indents is sometimes useful:
\setlength\parindent{0pt}

% Make margins smaller to fit more figures, tables etc on page: (optional)
\addtolength{\oddsidemargin}{-1.0in}
\addtolength{\evensidemargin}{-1.0in}
\addtolength{\textwidth}{2.0in}
\addtolength{\topmargin}{-0.8in}
\addtolength{\textheight}{1.6in}
% ==============================================================================

% ================================= EXAMPLES ===================================
\begin{comment}

% TABLE (TABULAR):
\begin{table}[]
\centering
  \begin{tabular}{|l|c|c|c|}
    \hline
    Colors: & Red    & Green  & Blue \\ \hline
    Red     & Red    & Yellow & Purple \\ \hline
    Green   & Yellow & Green  & Cyan \\ \hline
    Blue    & Purple & Cyan   & Blue \\ \hline
  \end{tabular}
  \caption{Caption}
  \label{tab:my_label}
\end{table}

% FIGURE:
The plot in figure \ref{fig:unique_name} clearly shows something important.
\begin{figure}[H]
\begin{center}
\includegraphics[width=0.3\textwidth]{figs/silicon.png}
\caption{The caption explains the figure.}
\label{fig:unique_name}
\end{center}
\end{figure}

% EQUATION:
\begin{equation}
\frac{df}{dt} = \lim_{h \to 0}\frac{f(t+h)-f(t)}{h}
\end{equation}

% ALIGNED MATH:
\begin{align*}
    r      &= \sqrt{x^2 + y^2 + z^2} & x &= r\cos \theta \sin \phi \\
    \theta &= \tan^-1(\frac{y}{x})   & y &= r\sin \theta \sin \phi \\
    \phi   &= \cos^-1(\frac{z}{r})   & z &= r\cos \phi
\end{align*}

% CENTERED DISPLAY MODE MATH:
\[ V_{th} = 0.45V \]

% INLINE MATH:
We used a supply voltage of 1.2 volts; \(V_{dd} = 1.2\text{V}\).

% MATRIX:
\[
\begin{Bmatrix} % Curly brackets
v_1 \\
v_2
\end{Bmatrix}
= \begin{bmatrix} % Square brackets
Z_{11} & Z_{12} \\
Z_{21} & Z_{22}
\end{bmatrix}
\begin{Bmatrix}
i_1 \\
i_2
\end{Bmatrix}
\]

% COLORS:
\definecolor{orange}{RGB}{255,127,0}
{\color{red} Red text.} {\color{orange} Orange text.}

% SOURCE CODE:
\inputminted{Python}{src/hello.py}

% MULTIPLE COLUMNS:
\begin{multicols}{2}
Write text and add figures etc. Content will be automatically split. You can
put figures, tables and even other multicols. Multiple columns can reduce
wasted space in a document.
\end{multicols}

% FOOTNOTES:
A simple footnote\footnote{Additional information} or
footnotemark\footnotemark{} is more readable.
\footnotetext{Footnotes are usefule for definitions, clarifications,
background info, e-mail/website info etc.}

\end{comment}
% ==============================================================================

% ================================= DOCUMENT ===================================
\begin{document}
    \renewcommand\marginsymbol[1][0pt]{%
  \tabto*{0cm}\makebox[-1cm][c]{$\mathbb{P}$}\tabto*{\TabPrevPos}}

\maketitle
\(\mathbb{P}\) marks the programming exercises,
we strongly recommend using the python programming language.
Exercises may be added/changed after publishing.

\section{Simple search algorithms}

Given the function \(f(x) = -x^4 + 2x^3 + 2x^2 - x\):

\subsection{Derivative}
What is its derivative \(f'(x)\) ?

\subsection{Plotting \marginsymbol}
Plot the function, and its gradient(derivative) from \(x=-2\) to \(x=3\).
Use python, wolfram alpha or another plotting tool of your choice.

\subsection{Gradient Ascent \marginsymbol}
\label{subsec:grada}
Maximize using gradient ascent.
You can try step size 0.1 and start somewhere in the range [-2, 3].
How does the choice of starting point and step size affect the algorithm's performance?
Is there a starting point where the algorithm would not even be able to find a local maximum?

\subsection{Exhaustive Search \marginsymbol}
\label{subsec:exhaust}
Assume that we are only interested in maxima of \(f(x)\),
where \(-2\leq x \leq 3\),
and \(x\) increases in steps of length 0.5 (\(\Delta x = 0.5\)).
Perform an exhaustive search to maximize \(f(x)\) and plot the result.

\subsection{Greedy Search and Hill Climbing}
In what way would greedy search and hill climbing differ for the maximization problem in Problem \ref{subsec:grada}?
Can you identify a starting position where the two algorithms might give different results?

\subsection{Possible improvements}
Gradient ascent, greedy search and hill climbing are quite similar, and are all based almost exclusively on exploitation.
Can you think of any additions to these algorithms in order to do more exploration?

\subsection{Exhaustive search vs. simulated annealing}
Which algorithm do you think is the most efficient at maximizing \(f(x)\) under the conditions in Problem \ref{subsec:exhaust};
exhaustive search or simulated annealing? Explain.
\section*{Corrections and suggestions}
Corrections of grammar, language, notation or suggestions for improving these exercises are appreciated.
E-mail me at: \href{mailto:olehelg@uio.no}{\textbf{olehelg@uio.no}} or use
\href{https://github.com/olehermanse/INF3490-AI_Machine_Learning}{\textbf{GitHub}}
to submit an issue or create a pull request.
\end{document}
% ==============================================================================
