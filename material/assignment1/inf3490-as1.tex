% ================================== HEADER ====================================
\documentclass{article}           % Sets style/look of many things.
% \documentclass{report}          % part, chapters, front page etc.
\usepackage{exsheets}
\usepackage[utf8]{inputenc}       % Encoding of input files UTF-8
\usepackage[T1]{fontenc}
\usepackage[scaled]{beramono}     % Font
\usepackage{color}                % Color text
\usepackage{titlesec}             % Select alternative section titles
\usepackage{fancyvrb}
\usepackage{verbatim}             % Comment environment
\usepackage{listings}             % Format and render text/code etc.
\usepackage{minted}               % Much better syntax highlighting
\usepackage{float}                % Control of floating environment/figure
\usepackage{graphicx,  subfigure} % Better figures, graphics, units etc.
\usepackage{multicol}             % Multiple columns
\usepackage{amsmath}              % Math: Equation, split, align etc.
\usepackage{siunitx}              % SI units
\usepackage{mathtools}            % Different math tools to use with amsmath
\usepackage{amssymb}              % Math symbols
\usepackage[
    colorlinks,
    citecolor=black,              % I like links with standard black color
    filecolor=black,
    linkcolor=black,
    urlcolor=black
]{hyperref}                       % Links in TOC etc.
\usepackage[all]{hypcap}          % Better links to floating environment

\usepackage{tabto}
\newcommand\marginsymbol[1][0pt]{%
  \tabto*{0cm}\makebox[\dimexpr-1cm-#1\relax][r]{$\mathbb{P}$}\tabto*{\TabPrevPos}}

\renewcommand{\thesubsection}{\thesection.\alph{subsection}}
\title{\vspace{-2cm}INF3490/INF4490 Mandatory Assignment 1:\\
Travelling Salesman Problem}
\author{Eivind Samuelsen, Jim Tørrresen, Magnus Olden, Ole Herman S. Elgesem}
\date{\today}

% Removing paragraph indents is sometimes useful:
%\setlength\parindent{0pt}

% Make margins smaller to fit more figures, tables etc on page: (optional)
\addtolength{\oddsidemargin}{-0.8in}
\addtolength{\evensidemargin}{-0.8in}
\addtolength{\textwidth}{1.6in}
\addtolength{\topmargin}{-0.8in}
\addtolength{\textheight}{1.6in}
% ==============================================================================

% ================================= DOCUMENT ===================================
\begin{document}
    \renewcommand\marginsymbol[1][0pt]{%
  \tabto*{0cm}\makebox[-1cm][c]{$\mathbb{P}$}\tabto*{\TabPrevPos}}

\maketitle
\section*{Rules}
Before you begin the exercise, review the rules at this website:
\begin{center}
\url{http://www.uio.no/english/studies/admin/compulsory-activities/mn-ifi-mandatory.html}
\end{center}
\section*{Delivering}
\textbf{Deadline:} 2016-09-23 23:59:00. All deliveries must be made through
\href{https://devilry.ifi.uio.no}{\textbf{Devilry}}.
\section*{What to deliver?}
Deliver one single zipped folder (.zip, .tgz or .tar.gz) which includes:
\begin{itemize}
    \item PDF report containing:
    \begin{itemize}
        \item Your name and username (!)
        \item Instructions on how to run your program.
        \item Answers to all questions from assignment.
        \item \emph{Brief} explanation of what you've done.
    \end{itemize}
    \item Source code
    \item The european cities file so the program will run right away.
    \item Any files needed for the group teacher to easily run your program on
          IFI linux machines.
\end{itemize}
\emph{Important: } if you weren't able to finish the assignment, use the PDF
report to elaborate on what you've tried and what problems you encountered.
Students who have made an effort and attempted all parts of the assignment
will get a second chance even if they fail initially. This exercise will be
graded PASS/FAIL.
\section*{Introduction}
In this exercise, you will attempt to solve an instance of the traveling
salesman problem (TSP) using different methods. The goal is to become familiar
with evolutionary algorithms and to appreciate their effectiveness on a
difficult search problem. You may use
whichever programming language you like, but we strongly suggest that you try
to use Python, since you will be required to write the second assignment in
Python. You must write your program from scratch (but you may use
non-EA-related libraries).

\section*{Problem}

The traveling salesman, wishing to disturb the residents of the major cities
in some region of the world in the shortest time possible, is faced with the
problem of finding the shortest tour among the cities. A tour is a path that
starts in one city, visits all of the other cities, and then returns to the
starting point. The relevant pieces of information, then, are the cities and
the distances between them. In this instance of the TSP, a number of European
cities are to be visited. Their relative distances are given in the data file
found at
\url{http://www.uio.no/studier/emner/matnat/ifi/INF3490/h16/assignment-1/european_cities.csv}.\\
(You will use permutations to represent tours in your programs. If you use
Python, the itertools module provides a permutations function that returns
successive permutations, this is useful for exhaustive search)

\begin{table}[]
\centering
  \begin{tabular}{|c|c|c|c|c|c|c|}
    \hline
              & Barcelona & Belgrade & Berlin  & Brussels & Bucharest & Budapest \\ \hline
    Barcelona & 0         & 1528.13  & 1497.61 & 1062.89  & 1968.42   & 1498.79  \\ \hline
    Belgrade  & 1528.13   & 0        & 999.25  & 1372.59  & 447.34    & 316.41   \\ \hline
    Berlin    & 1497.61   & 999.25   & 0       & 651.62   & 1293.40   & 689.06   \\ \hline
    Brussels  & 1062.89   & 1372.59  & 651.62  & 0        & 1769.69   & 1131.52  \\ \hline
    Bucharest & 1968.42   & 447.34   & 1293.40 & 1769.69  & 0         & 639.77   \\ \hline
    Budapest  & 1498.79   & 316.41   & 689.06  & 1131.52  & 639.77    & 0        \\ \hline
  \end{tabular}
  \caption{First 6 cities from csv file.}
  \label{tab:cities}
\end{table}

\section*{Exhaustive Search}
First, try to solve the problem by inspecting every possible tour. Start by
writing a program to find the shortest tour among a subset of the cities (say,
6 of them). Measure the amount of time your program takes. Incrementally add
more cities and observe how the time increases.\\

What is the shortest tour (i.e., the actual sequence of cities, and its
length) among the first 10 cities (that is, the cities starting with B,C,D,H
and I)? How long did your program take to find it? How long would you expect
it to take with all 24 cities?

\section*{Hill Climbing}
Then, write a simple hill climber to solve the TSP. How well does the hill
climber perform, compared to the result from the exhaustive search for the
first 10 cities? Since you are dealing with a stochastic algorithm, you should
run the algorithm several times to measure its performance. Report the length
of the tour of the best, worst and mean of 20 runs (with random starting
tours), as well as the
\href{https://en.wikipedia.org/wiki/Standard_deviation}{\textbf{standard deviation}}
of the runs, both with the 10 first cities, and with all 24 cities.

\section*{Genetic Algorithm}
Next, write a genetic algorithm (GA) to solve the problem. Choose mutation
and crossover operators that are appropriate for the problem (see chapter 3
of the Eiben and Smith textbook). Choose three different values for the
population size. Define and tune other parameters yourself and make
assumptions as necessary (and report them, of course).\\

For all three variants: As with the hill climber, report best, worst, average
and deviation of tour length out of 20 runs of the algorithm (of the best
individual of last generation). Also, find and plot the average fitness of
the best fit individual in each generation (average across runs), and include
a figure with all three curves in the same plot in the report. Conclude which
is best in terms of tour length and number of generations of evolution time.\\

Among the first 10 cities, did your GA find the shortest tour (as found by the
exhaustive search)? Did it come close? For both 10 and 24 cities: How did the
running time of your GA compare to that of the exhaustive search? How many
tours were inspected by your GA as compared to by the exhaustive search?

\section*{Hybrid Algorithm (INF4490 only)}
Implement a hybrid algorithms to solve the TSP: Couple your GA and hill
climber by running the hill climber a number of iterations on each individual
in the population as part of the evaluation. Test both Lamarckian and
Baldwinian learning models and report the results of both variants in the same
way as with the pure GA (min, max, mean and std.dev. of the end result and an
averaged generational plot). How does the results compare to that of the pure
GA, considering the number of evaluations done?

\section*{Corrections and suggestions}
Corrections of grammar, language, notation or suggestions for improving this assignment are appreciated.
E-mail me at: \href{mailto:olehelg@uio.no}{\textbf{olehelg@uio.no}} or use
\href{https://github.com/olehermanse/INF3490-AI_Machine_Learning}{\textbf{GitHub}}
to submit an issue or create a pull request.
\end{document}
% ==============================================================================
